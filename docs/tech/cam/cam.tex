 %%%%%%%%%%%%%%%%%%%%%%%%%%%%%%%%%%%%%%%%%%%%%%%%%%%%%%%%%%%%%%%%%%%%%%
\documentclass[oneside,branding,toc,article]{satdoc}
\usepackage{satbib}
\usepackage{a4,url,epsfig,amssymb,amsbsy,color,times}
\usepackage[ps2pdf]{hyperref}
\hypersetup{
  pdfauthor = {Jan Paral},
  pdftitle = {CAM},
  pdfsubject = {Hybrid Code},
  pdfkeywords = {hybrid code, plasma},
  pdfcreator = {LaTeX with hyperref package},
  pdfproducer = {dvips + ps2pdf},
  bookmarks=true,
  bookmarksopen=true,
  bookmarksopenlevel=1,
  bookmarksnumbered=true,
  breaklinks=true,
  colorlinks=true,
  linkcolor=blue,
  anchorcolor=blue,
  citecolor=blue
}

\newcommand{\mbf}[1]{\mathbf{#1}}              % bf in math mode: vectors
\newcommand{\trm}[1]{\textrm{#1}}              % normal font in math mode
\newcommand{\mrm}[1]{\mathrm{#1}}              % normal font in math mode
\newcommand{\udx}[1]{ \, \mathrm{d}#1}         % dx after integral: \udx{x}

% \doublespacing

\title{CAM Hybrid code}

\SATauthor{Jan Paral}{jparal@phys.ualberta.ca}
\SATmodification{2008/01/21}
\SATdocversion{1.0}
\SATaddrevision{2008/01/21}{1.0}{Initial version}

\begin{document}
\maketitle

\section{Introduction}

As computer resources available to researchers increases, numerical simulations
are becoming more and more important as a tool standing between theoretical
results and pure experiment.  We can compare the theoretical results of the
research with the computational simulations before investing enormous money
into the equipment needed to perform the experiment.  This is true especially
in space physics where budget to build a spacecraft can reach millions of
dollars.

\begin{figure}[!hbp]
  \centering
  \includegraphics[width=11cm]{pic/codesnfo.eps.gz}
  \caption{Plasma codes.}
  \label{fig:nfo}
\end{figure}

Several different techniques how to simulate the plasma are available to the
researches, depending on the physical problem, temporal or spatial scales they
need to resolve.  Two of the most common techniques are particle-in-cell (PIC)
and magnetohydrodynamics (MHD) codes summarized in Fig.  \ref{fig:nfo}.  The
former one (PIC) uses a single point in the phase-space representing a cloud of
particles with similar velocity at similar position.  This particle approach is
used for ions as well as electrons which enforce the limit on a range of time
steps $\triangle t$ available for particle advancing (i.e. $\triangle t <
\Omega_e^{-1}$) where $\Omega_e$ is electron gyrofrequency.  The second
approach is to approximate plasma by fluid and couple the system with Maxwell's
equations through the force acting on a single fluid element by Lorentz
force. This system is called magnetohydrodynamics.

In many cases, ions play an important role in studied processes and we can
easily neglect electrons. The codes where different component (specie) of the
plasma is treated by different technique is called hybrid code. The most common
example is where ions are treated as PIC and electrons as a charge neutralizing
fluid treated by MHD formalism.  For several different approximations you can
see \cite{lipatov02} which gives good overview of today status of the hybrid
plasma simulation technology.

\section{Hybrid code}

Several different techniques are used to advance the hybrid code into the next
time level (i.e. \cite{winske85, wiom93, wiom96}).  We choose to use Current
Advance Method and Cyclic Leapfrog (CAM-CL) published by \cite{matthews94}
which has the advantage of going through the particles only once instead of
pre-pushing of particles in order to obtain electric field \citep{harned82}.
The various implementations of the codes employ different numerical techniques:
\cite{harned82} is using a predictor-corrector scheme (see \cite{nr92});
\cite{wiqu88} using a moment method or \cite{hoshan89} substepping magnetic
field.  The CAM-CL implementation is using advantage of field sub-stepping
instead of using the same step as for particles.  Recently, this code was
expanded into 3D by \cite{travea07} and used to explore magnetosphere of
Mercury.

There are few differences when compared to other hybrid codes:
\begin{itemize}
\item Multiple ion species are treated in a single computational pass through
  data arrays.
\item CAM advances ionic current density instead of fluid velocity.
\item Velocity is collected a half time step ahead, before equation of motion
  is applied.
\item Magnetic field is sub-stepped using modified midpoint method \citep{nr92}
  for better time resolution and to prevent dispersion.
\end{itemize}

The system is governed by Vlasov-fluid equations:

\begin{equation}
  \frac{\, {\mathrm d}{\mathbf{x}_s}}{\, {\mathrm d}{t}} = \mathbf{v}_s
\end{equation}

\begin{equation}
  \frac{\, {\mathrm d}{\mathbf{v}_s}}{\, {\mathrm d}{t}} =
  \frac{q_s}{m_s} ( \mathbf{E} + \mathbf{v}_s \times \mathbf{B} )
\end{equation}

\begin{equation}
  \frac{\partial \mathbf{B}}{\partial t} = - \nabla \times \mathbf{E}
  \label{eq:faraday}
\end{equation}

\begin{equation}
  \nabla \times \mathbf{B} = \mu_0 \mathbf{J}
  \label{eq:ampere}
\end{equation}

\begin{equation}
  n_e m_e \frac{\, {\mathrm d}{\mathbf{u}_e}}{\, {\mathrm d}{t}} =
  - n_e e \mathbf{E} + \mathbf{J}_e \times \mathbf{B} - \nabla p_e
  \label{eq:fluid}
\end{equation}

\begin{equation}
  p_e = n_e k_B T_e
  \label{eq:eos}
\end{equation}
where the symbols mean: ion position $\mbf{x}_s$; ion velocity $\mbf{v}_s$; ion
mass $m_s$; ion charge $q_s$; electric field $\mbf{E}$; magnetic field
$\mbf{B}$; magnetic permeability $\mu_0$; current density $\mbf{J}$; electron
number density $n_e$; electron mass $m_e$; electron fluid velocity $\mbf{u}_e$;
magnitude of electronic charge $e$; electronic current density $\mbf{J}_e = -
n_e e \mbf{u}_e$; electron fluid pressure $p_e$; Boltzmann's constant $k_B$;
and electron temperature $T_e$. The subscript $s$ refers to the ion specie. We
are using Darwin's approximation so the displacement current is neglected in
Maxwell's equation \ref{eq:ampere} and we are assuming massless electron fluid
so lefthand term of equation \ref{eq:fluid} can be neglected. By adding a term
$\eta \mathbf{J_e}$ on the right-side of equation \ref{eq:fluid}, where $\eta$
is resistivity of plasma, we can introduce an artificial resistivity into the
system which cause the damping of high frequency waves.  In the real situation,
Eq. \ref{eq:eos} describing isotropic and isothermal plasma is not valid and we
have to use rather adiabatic approximation where we substitute electron
temperature $T_e$ by $T_e = T_{e0} (n_e / n_{e0})^{\gamma - 1}$, where $\gamma$
is an adiabatic index.

In our case the electric field is not time dependent and can be evaluated as a
function of current density, magnetic field, charge density and electron
pressure when we combine \ref{eq:fluid}, \ref{eq:ampere}, and substitute
electron current density by $\mbf{J} = \mbf{J}_e + \mbf{J}_i$ and charge
density by $\rho_c = n_s q_s$
\begin{equation}
  \label{eq:e}
  \mbf{E} = - \frac{\mbf{J}_i \times \mbf{B}}{\rho_c} +
  \frac{(\nabla \times \mbf{B}) \times \mbf{B}}{\mu_0 \rho_c} -
  \frac{\nabla p_e}{\rho_c}
\end{equation}
substituting into equation \ref{eq:faraday} we obtain relation for advancing of
magnetic field given by
\begin{equation}
  \label{eq:b}
  \frac{\partial \mbf{B}}{\partial t} =
  \nabla \times \frac{\mbf{J}_i \times \mbf{B}}{\rho_c} -
  \nabla \times \frac{(\nabla \times \mbf{B}) \times \mbf{B}}{\mu_o \rho_c}
\end{equation}

Then all other variables are defined as
\begin{equation}
  n_s = \int f_s(\mathbf{x}_s,\mathbf{v}_s) ~ \, \mathrm{d\mathbf{v}_s}
\end{equation}
is the density and $f$ is distribution function. Bulk velocity:
\begin{equation}
  \mathbf{u}_s =
  \frac{1}{n_s} \int \mathbf{v}_s f_s(\mathbf{x}_s,\mathbf{v}_s) \, \mathrm{d\mathbf{v}_s}
\end{equation}
\begin{equation}
  \rho_c = \sum_{s}n_sq_s
\end{equation}
\begin{equation}
  \rho_m = \sum_{s}n_sm_s
\end{equation}
\begin{equation}
  \mathbf{J}_s = q_sn_s\mathbf{u}_s
\end{equation}
\begin{equation}
  \mathbf{J}_i = \sum_{s}\mathbf{J}_s
\end{equation}
\begin{equation}
  \mathbf{J} = \mathbf{J}_i + \mathbf{J}_e
\end{equation}

\section{Particle advancing}
We keep ion velocity $\mathbf{v}_s$ and its position $\mathbf{x}_s$ in the
different time levels $t$ and $t+1/2 \triangle t$ respectively so we can employ
leapfrog algorithm as follows: First, we evaluate velocity half timestep ahead
to obtain best estimate for velocity
\begin{equation}
  \label{eq:vh}
  \mathbf{v}_s^{1/2} = \mathbf{v}_s^{0} + \frac{\triangle t}{2} \frac{q}{m}
  \left( \mathbf{E}^{1/2} + \mathbf{v}_s^{0} \times \mathbf{B}^{1/2} \right)
\end{equation}
and then evaluate entire step by using estimate of velocity
$\mathbf{v}_s^{1/2}$ as follows
\begin{equation}
  \label{eq:vf}
  \mathbf{v}_s^{1} = \mathbf{v}_s^{0} + \triangle t \frac{q}{m}
  \left( \mathbf{E}^{1/2} + \mathbf{v}_s^{1/2} \times \mathbf{B}^{1/2} \right)
\end{equation}
when the velocity is known in time step 1 we can simply advance the position of
particle as
\begin{equation}
  \label{eq:x}
  \mathbf{x}_s^{3/4} = \mathbf{x}_s^{1/2} + \triangle t \mathbf{v}_s^{1}
\end{equation}
where $\mathbf{v}_s^{1}$ is already known from previous step.

\section{Magnetic field and current density advancing}
Magnetic field is advanced using equation \ref{eq:b}. We apply the same
leap-frog technique as for velocity advancing described in \cite{nr92} which is
a modified midpoint method \citep{nr92}.  We keep two copies of magnetic field
and leapfrog one copy over the other.  We substep magnetic field with time step
$\triangle t_B = \triangle t / n$ as a fraction of time step for particles
advancing.  This allow us to resolve even high frequency changes in the
magnetic field.  Because magnetic field is a function of electric field we need
to evaluate electric field in each step. Then magnetic field will be advanced
as
\begin{eqnarray}
  B_1 & = & B_0 - \triangle t_B \nabla \times E_0 \nonumber \\
  B_2 & = & B_1 - 2 \triangle t_B \nabla \times E_1 \nonumber \\
  ... \nonumber \\
  B_{p+1} & = & B_{p-1} - 2 \triangle t_B \nabla \times E_p \nonumber \\
  ... \nonumber \\
  B_{n} & = & B_{n-2} - 2 \triangle t_B \nabla \times E_{n-1} \nonumber \\
  B_{p}^* & = & B_{n-1} - \triangle t_B \nabla \times E_{n}
\end{eqnarray}
where odd subscript together with $B^*$ represents one copy of the field and
even subscript represent the second copy of the magnetic field.  On the end of
field advancing we can average two copies of the field or use the difference
between $B^*$ and $B_n$ as a measure of the error and change number of substeps
accordingly.

Because we need to evaluate electric field during the processes of advancing as
\begin{equation}
  E_p = E (\rho_c^{1/2}, J_i^{1/2}, B_p, T_e)
\end{equation}
we require current density $J_i$ to be known at time level 1/2. Velocity of
particles is not known at time level 1/2, we have to approximate current
density by advancing it from $J_i^* = J (v_i^{0}, x_i^{1/2})$ into $J_i^{1/2}$
by using moment equation:
\begin{equation}
  \label{eq:curradv}
  J_i^{1/2} = J_i^* = \frac{\triangle t}{2} (\Lambda E^* + \Gamma B^{1/2})
\end{equation}
where $\Lambda$ and $\Gamma$ are defined as follows:
\begin{equation}
  \label{eq:lambda}
  \Lambda = \sum_s \phi_{sj}^{1/2} \frac{q_s^2}{m_s}
\end{equation}
\begin{equation}
  \label{eq:lambda}
  \Gamma = \sum_s \phi_{sj}^{1/2} \frac{q_s^2}{m_s} v_x^0
\end{equation}
and $\phi$ is a weighting function $\phi (\mathbf{x}_s)$.
\begin{figure}[!h]
  \centering
  \includegraphics[width=12cm]{pic/weight.eps.gz}
  \caption{Example of weighting functions a) delta function, b) linear
    weighting function. Courtesy by \cite{bila85}.}
  \label{fig:weight}
\end{figure}
Weighting function is, in the most cases, a linear basis function as in
Fig. \ref{fig:weight}b, prescribed by
\begin{eqnarray}
  \label{eq:weight}
  \phi (0,0) & = & (1-x)(1-y) \nonumber \\
  \phi (1,0) & = & x(1-y)     \nonumber \\
  \phi (1,1) & = & xy         \nonumber \\
  \phi (0,1) & = & (1-x)y
  \end{eqnarray}
in the case of 2D, where $x$ and $y$ are normalized distance $(x,y) \in [0,1]
\times [0,1]$ and parameter 0 and 1 represent the vertex of the computational
cell.

\section{Main Loop}

In Fig. \ref{fig:loop} you can see the steps described above assembled
together.
\begin{figure}[!h]
  \centering
  \includegraphics[height=11cm,width=11cm]{pic/camsch.eps.gz}
  \caption{Cartoon representing loop of CAM-CL code. Courtesy by
    \cite{matthews94}.}
  \label{fig:loop}
\end{figure}
To advance simulation into next time step we need to perform three steps.  Lets
assume we are given these variables at the beginning of the simulation:
$x^{1/2}$, $v^0$, $\rho_c^{0}$, $\rho_c^{1/2}$, $J_i^0$, $\Lambda$, $\Gamma$
and $J_i^+ = J_i^* (x^{1/2},v^0)$ where electron temperature $T_e$ is constant.

Advance magnetic field $B^0$ into $B^{1/2}$, make an estimate of current
density by advancing $J_i^*$ into $J_i^{1/2}$ to get electric field $E^{1/2}$

\begin{eqnarray}
  \label{eq:step1}
  B^{1/2} & = & B^0 - \int_0^{\triangle t/2} \nabla \times E(\rho_c^0, J_i^0,
  B(t), T_e) dt \nonumber \\
  E^* & = & E(\rho_c^{1/2}, J_i^0, B^{1/2}, T_e) \nonumber \\
  J_i^{1/2} & = & J_i^* + \frac{\triangle t}{2}(\Lambda E^* + \Gamma
  \times B^{1/2}) \nonumber \\
  E^{1/2} & = & E(\rho_c^{1/2}, J_i^{1/2}, B^{1/2}, T_e) \nonumber
\end{eqnarray}

Advance particles ($v^1$, $x^{3/2}$) and collect moments of distribution
function.

\begin{eqnarray}
  \label{eq:step2a}
  v_s^{1/2} & = & v_s^{0} + \frac{\triangle t}{2} \frac{q}{m}
  (E^{1/2} + v_s^{0} \times B^{1/2}) \nonumber \\
  v_s^{1} & = & v_s^{0} + \triangle t \frac{q}{m}(E^{1/2} + v_s^{1/2} \times
  B^{1/2}) \nonumber \\
  x_s^{3/4} & = & x_s^{1/2} + \triangle t v_s^{1} \nonumber
\end{eqnarray}

\begin{eqnarray}
  \label{eq:step2b}
  \rho_c^{3/2} & = & \rho_c (x^{3/2}) \nonumber \\
  J_i^- & = & J_i^* (x^{1/2}, v^1) \nonumber \\
  J_i^+ & = & J_i^* (x^{3/2}, v^1) \nonumber \\
  \Gamma & = & \Gamma (x^{3/2}, v^1) \nonumber \\
  \Lambda & = & \Lambda (x^{3/2}, v^1) \nonumber
\end{eqnarray}

From these variables we can easily obtain $\rho_c^{1}$ and $J_i^{1}$ by taking
average and finish advancing of magnetic field into time level $B^1$

\begin{eqnarray}
  \label{eq:step2b}
  \rho_c^{1} & = & \frac{1}{2} (\rho_c^{1/2} + \rho_c^{3/2}) \nonumber \\
  J_i^{1} & = & \frac{1}{2} (J_i^- + J_i^+) \nonumber \\
  B^{1} & = & B^{1/2} - \int_{\triangle t/2}^{\triangle t} \nabla \times
  E(\rho_c^{1}, J_i^1, B(t), T_e) dt \nonumber
\end{eqnarray}

Because we expected the position at the beginning to be already at $x^{1/2}$ we
have to perform first and the last step before and after any output of
variables as follows:

\begin{eqnarray}
  \label{eq:first}
  \rho_c^{0} & = & \rho_c (x^{0}) \nonumber \\
  J_i^0 & = & J_i^* (x^{0}, v^0) \nonumber \\
  x_s^{1/2} & = & x_s^{0} + \frac{\triangle t}{2} v_s^{0} \nonumber \\
  \rho_c^{1/2} & = & \rho_c (x^{1/2}) \nonumber \\
  J_i^+ & = & J_i^* (x^{1/2}, v^0) \nonumber \\
  \Gamma & = & \Gamma (x^{1/2}, v^0) \nonumber \\
  \Lambda & = & \Lambda (x^{1/2}, v^0) \nonumber
\end{eqnarray}

and last step
\begin{eqnarray}
  \label{eq:last}
  x_s^{1} & = & x_s^{3/2} - \frac{\triangle t}{2} v_s^{1} \nonumber \\
  \rho_m^{1} & = & \rho_m (x^{1}) \nonumber \\
  u_i^{1} & = & u_i (x^{1}, v^1) \nonumber
\end{eqnarray}

\section{Model Scaling}
For better scaling of the physical problems we use a scaling called
\emph{hybrid units}. We usually take protons of the solar wind as a reference
to express other species and variables so hybrid units are dimensionless. If we
mark all simulation hybrid units by index $H$ and physical units by $SI$ then
each variable $x$ can be expressed $[x]_{\rm H} = [x]_{\rm SI} / u$ where $u$
is physical unit. Then when we adopt mass of the proton $m_p$ as unit of mass
and charge $e$ then we can define \emph{hybrid units}: unit of magnetic field
$B_{\rm SW}$ (magnetic field of solar wind); unit of speed $v_A$ Alfv\'en
speed; unit of time $\Omega_{p \rm SW}$ (solar wind proton cyclotron time);
unit of length $\Lambda_{p \rm SW} = c/\omega_{pi} = v_A / \Omega_{p \rm SW}$
(inertial length); unit of charge density $n_{p \rm SW} e$; unit of electric
field $v_A B_{\rm SW}$; unit of energy $\rho_m (c/\omega_{p \rm SW})^3 v_A^2$
where $c$ is the speed of light, $\Omega_{p \rm SW} = e B_{\rm SW} / m_p$ is
solar wind proton gyrofrequency, and $\omega_{pi}^2 = n_0 e^2 / \epsilon_0 m_p$
is ion plasma frequency. Note that in hybrid units we keep magnetic
permeability $\mu_0 = 1$, ion beta
\begin{equation}
  \label{eq:betai}
  \beta_s = \frac{p_s}{p_B} = \frac{v_{th,s}^2 ~ \rho_{m,s}}{v_A^2 ~ \rho_m}
\end{equation}
electron beta $\beta_e = 2 \tau_e$, and speed of sound
\begin{equation}
  \label{eq:cs}
  c_s^2 = \frac{p_i + p_e}{\rho_m} = \frac{1}{2}(\beta_i + \beta_e)^2 v_A^2
\end{equation}
where $\tau_e = k_B T_e/e$ is a measure of the electron temperature. Alv\'en
speed $v_A$ is defined as
\begin{equation}
  \label{eq:va}
  v_A^2 = \frac{B^2}{\mu \rho_m}
\end{equation}

\section{Shock wave simulations}
To demonstrate the code we implemented 1D quasi-perpendicular shock wave. 
\begin{figure}[!h]
  \centering
  \includegraphics[width=13cm]{pic/shock.eps.gz}
  \caption{Cartoon of the shock wave formed by interaction of solar wind with
    magnetic obstacle of the Earth. Courtesy of \cite{bt96}.}
  \label{fig:shock}
\end{figure}
The shock wave is a discontinuity (see Fig. \ref{fig:shock}) formed when solar
wind slow down from super-magnetosonic to sub-magnetosonic speed due to the
magnetic obstacle (i.e. Earth). The discontinuity is represented by a change in
particle density and magnetic field.  Depending on the angle between
interplanetary magnetic field (IMF) and normal to the surface of the shock wave
we can categorize a shock wave profile as: parallel, perpendicular and oblique.
Shock waves research was summarized in \cite{kennel85} followed by many others
studies \cite{winske85, thomas89, hema97}.

To initialize simulation we load Maxwell-Boltzmann distribution of particles
drifting in positive $x$ direction with bulk velocity $\mbf{u}$ and we keep
injecting particles from the left boundary during the entire simulation.  Then,
we reflect all particles which reach the right boundary of simulation box by
changing the sign of $v_x$ component of its velocity. Because magnetic field is
directly computed from electric field, we need to specify only boundary
conditions for electric field. We keep the electric field constant at left
boundary by imposing $\mathbf{E} = - \mathbf{u} \times \mathbf{B}$ and we set
the tangential component of electric field to 0 at the right boundary.  Note
that we can speed up the entire process of building up of shock wave by loading
$\tanh$-like profile of number density and magnetic field in the
simulation box during initialization process.

\begin{figure}[!h]
  \centering
  \includegraphics[width=15cm]{pic/cfgfile.eps.gz}
  \caption{Example of configuration file for shock wave simulation by hybrid
    code.}
  \label{fig:cfg}
\end{figure}

In Fig. \ref{fig:cfg} you can see an example of initialization we used to
simulate shock wave by hybrid code.  First three parameters describe simulation
time \texttt{time.t0} is a starting time of simulation, where we expect to
implement a restarting mechanism of simulation from the later times in the near
future.  Parameter \texttt{time.max} and \texttt{time.dt} represent maximal
time and time step for the particles in units of ion cyclotron period
$\Omega_i^{-1}$.  Magnetic field is sub-stepped with 10 times smaller time step
(\texttt{field.nsub}) as compared to the time step used for particles.  We
build mesh with 500 cells in $x$ direction (\texttt{grid.ncg}) with resolution
0.15 (\texttt{grid.dl}) of inertial length $L_{in}$.  At the beginning of the
simulation we load 70 protons (\texttt{np}) into each of the cells with
Maxwell-Boltzmann distribution described by parameters $\beta_e = 0.5$
(\texttt{betae}), $\beta_i = 0.5$ (\texttt{betas}), and temperature anisotropy
$T_\perp / T_\parallel = 1$ (\texttt{rmds}).  Entire distribution is drifting
with $u_x = 3 v_A$ in $x$ direction (\texttt{vxs}).  Artificial resistivity,
included by an extra term $\eta J$ in general Ohm's law represented by
Eq. \ref{eq:e}, is set to be 0.001 by parameter \texttt{resist}.  The external
magnetic filed forms 90$^\circ$ with $x$ axis, initialized by parameters
\texttt{phi} and \texttt{psi} defining magnetic field as $B_x = \cos (\phi)$,
$B_y = \sin (\phi) \cos (\psi)$ and $B_z = \sin (\psi)$.

\begin{figure}[!h]
  \centering
  \includegraphics[height=12cm,angle=90]{fig/bft_c3.eps.gz}
  \caption{Magnetic field magnitude as measured by CLUSTER II/SC3 during the
    crossing of quasi-parallel shock wave.}
  \label{fig:cluster}
\end{figure}
\begin{figure}[!h]
  \centering
  \includegraphics[width=10cm]{pic/shocks.eps.gz}
  \caption{Typical profiles of shock wave for different orientation of magnetic
    field. Courtesy by \cite{bt96}. }
  \label{fig:shocks}
\end{figure}

These initial parameters and boundary conditions form a parallel shock wave
travelling in $-x$ direction in the frame reference of the simulation box. As
you can see in Fig. \ref{fig:shocks} adopted from \cite{bt96}, the shock
structure is very clearly separated into the three regions: 1) upstream in the
solar wind, 2) transition region with high gradient discontinuity in density;
and 3) downstream.  To compare with real measurement we plotted the real
measurements taken by third satellite of CLUSTER II mission during passing
through quasi-parallel shock wave on Fig. \ref{fig:cluster}, plotting magnetic
field magnitude versus time with very similar parameters of solar wind and
interplanetary magnetic field we used for our simulation.

We took the output from out simulation at time $t=8 \Omega_i^{-1}$, when the
shock front is formed properly, and plot distribution function of particles in
$x$ direction and the first moment (i.e. bulk velocity $u_x$) and plot those
variables a function of spatial $x$ axis in Fig. \ref{fig:dist}.  You can see
that particles are drifting with the mean velocity of $3 v_A$ in the $x$
direction in the upstream region and they are rapidly slowed down and partially
reflected back in the transition region.  Particle density and $y$ component of
magnetic field in Fig. \ref{fig:dnby} form a discontinuity in transition region
where magnetic field jumps approximately three times of its value in the
upstream which is in the agreement with experimental results in
Fig. \ref{fig:cluster}.

In Fig. \ref{fig:bref} you can compare magnetic field evolution for two
different study cases of ion beta $\beta_i$ as a function of time.  In the
upper plot for $\beta_p = 0.2$ and in the lower plot for $\beta_p = 1$.  You
can clearly see a phenomenon called reformation where the part of the shock
front is teared from the transition region and travel faster then the rest of
the shock wave.  Note that results in Fig. \ref{fig:bref} are not from our
simulation code but from the different implementation of the CAM-CL hybrid code
developed by Pavel Travnicek.  Results from these simulation will serve us as a
reference in future.

\begin{figure}[!h]
  \centering
  \includegraphics[width=14cm]{fig/pclux.eps.gz}
  \includegraphics[width=14cm]{fig/ux.eps.gz}
  \caption{$x$ direction of particle distribution function in upper plot and
    first moment (i.e. bulk velocity) in lower plot. }
  \label{fig:dist}
\end{figure}

\begin{figure}[!h]
  \centering
  \includegraphics[width=14cm]{fig/dn.eps.gz}
  \includegraphics[width=14cm]{fig/by.eps.gz}
  \caption{Density profile in the units of solar wind density $n_{sw}$ in upper
    plot and magnetic field $y$ component in the units of solar wind magnetic
    field $B_{sw}$}
  \label{fig:dnby}
\end{figure}

\begin{figure}[!h]
  \centering
  \includegraphics[width=12cm]{fig/brefb02m40.eps.gz}
  \includegraphics[width=12cm]{fig/brefb10m40.eps.gz}
  \caption{Evolution of magnetic field in time (y-axis) for mach number $M_A =
    4$ and two various betas: upper plot $\beta_p = 0.2$, lower plot $\beta_p =
    1$. Colour scale represents magnitude of magnetic field in units of
    $B_{sw}$. }
  \label{fig:bref}
\end{figure}

\section{Conclusions}
Several aspects still need to be addressed.  In the most of the outputs you can
clearly see the output variables to be very noisy which is a direct result of
technique how the particles are described (i.e. discrete points in the phase
space).  We can prevent this phenomenon by further smoothing of electric field
and moments of distribution function.  Another possible solution would be to
implement better description of a single particle by introducing a shape
function as described in \cite{coppea96}, which uses ``blobs'' giving the
particle the shape in the phase space.

Shock waves posses very sharp transition region in particle density and
magnetic field.  To resolve better the transition region we need to increase
spatial resolution.  In the case of 2D or even 3D simulation we need to employ
more sophisticated methods.  The possible solution is using Adaptive Mesh
Refinement Method (AMR) which increase the spatial resolution by creating
sub-mesh with higher resolution in the regions of interest (i.e. high
gradient).

When expanding the code into 2D or even 3D a parallel computer is needed
because number of particles grow as $O(n^d)$, where $n$ is spatial resolution
and $d$ is dimensionality of the code.  The most suitable choice would be
Message Passing Interface (MPI) which is a standard library supporting
communication in the parallel environment of clusters or shared memory
computers as well.

%%%%%%%%%%%%%%%%%%%%%%%%%%%%%%%%%%%%%%%%%%%%%%%%%%%%%%%%%%%%%%%%%%%%%%
%%% References
%%%%%%%%%%%%%%%%%%%%%%%%%%%%%%%%%%%%%%%%%%%%%%%%%%%%%%%%%%%%%%%%%%%%%%
\bibliographystyle{satbib}
\bibliography{articles}
%%%%%%%%%%%%%%%%%%%%%%%%%%%%%%%%%%%%%%%%%%%%%%%%%%%%%%%%%%%%%%%%%%%%%% 
\end{document}
